\subsubsection*{Prevention Model}
The goal of the prevention model is to provide an accessible tool for the early diagnosis of heart 
diseases, potentially usable by non-experts. Therefore, it is crucial to develop a model that 
minimizes the number of false normal predictions to accurately indicate the presence or not
of disease or identify artifacts in the provided data.

To achieve this, different heart diseases were grouped together, transforming the problem 
into a 3-class classification task: normal, disease, and artifact. Grouping the diseases not 
only simplified the classification but also balanced the class distribution. The data was 
divided into training and testing sets in an 80-20 ratio, and various models were evaluated, 
as shown in Table \ref{tab:models}.

The primary metrics for evaluating the models were the ROC-AUC score, false positive rate (FPR), 
and true positive rate (TPR), with F1-score and accuracy as secondary metrics. 
To adapt binary metrics for multi-class classification, the one-vs-rest strategy was employed. 
Specifically, we focused on the normal-vs-rest case to minimize false normal predictions.

In summary, each model was trained on the 3-class classification problem but was evaluated based on 
its binary classification performance (normal-vs-rest). The best model was selected based on its 
ROC-AUC score and performance at specific FPR levels (1\%, 5\%, 10\%, and 20\%). The objective was 
to minimize false normal predictions while maximizing true normals. A model predicting no cases 
as normal to achieve a 0\% FPR would be ineffective.