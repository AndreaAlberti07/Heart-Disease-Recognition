
\subsubsection*{Experimented Architectures}
\rowcolors{2}{blue!8}{blue!18}
\begin{table}[h!]
    \centering
    \footnotesize
    \begin{tabular}{|l|l|}
        \hline
        \textbf{Name}           & \textbf{Architecture (hidden layers)}                                 \\ \hline
        Random Forest                 & -                                                     \\ \hline
        XGBoost                 & -                                                     \\ \hline
        CatBoost                & -                                                     \\ \hline
        LightGBM                & -                                                     \\ \hline
        MLP\_Basic              & (128, 64, 32)                                         \\ \hline
        MLP\_Ultra              & (512, 256, 128, 64, 32)                               \\ \hline
        MLP\_Large              & (256, 128, 64, 32)                                    \\ \hline
        MLP\_Small              & (64, 32)                                              \\ \hline
        MLP\_Tiny               & (32, 16)                                              \\ \hline
        MLP\_Reverse            & (32, 64, 128, 256, 512, 256, 128, 64, 32)             \\ \hline
        MLP\_Bottleneck         & (512, 64, 32)                                         \\ \hline
        MLP\_Rollercoaster      & (512, 128, 256, 128, 256, 64, 32)                     \\ \hline
        MLP\_Hourglass          & (512, 256, 128, 64, 32, 64, 128, 256, 512)            \\ \hline
        MLP\_Pyramid            & (1024, 512, 256, 128, 128, 128, 64, 32)               \\ \hline
        MLP\_Wide               & (1024, 1024)                                          \\ \hline
        MLP\_WideUltra          & (1024, 1024, 128, 32)                                 \\ \hline
        MLP\_Sparse             & (32, 16, 8)                                           \\ \hline
        MLP\_Dropout            & (128, 64, 32)                                         \\ \hline
        MLP\_Ensemble1          & MLP\_Basic, Large, Ultra                    \\ \hline
        MLP\_Ensemble2          & RandomForest, MLP\_Ultra                              \\ \hline
        MLP\_Ensemble3          & MLP\_Rollercoaster, Large                        \\ \hline
        MLP\_Ensemble4          & MLP\_Rollercoaster, Large, Ultra            \\ \hline
        MLP\_Ensemble5          & RandomForest, MLP\_Ultra, Rollercoaster          \\ \hline
        MLP\_Ensemble6          & MLP\_Rollercoaster, Large, Ultra, Wide \\ \hline
        ALL\_Ensemble           & All models majority vote                              \\ \hline
        CB\_ALL\_Ensemble       & All models CatBoost                                   \\ \hline
    \end{tabular}
    \caption{Models names and architectures.}
    \label{tab:models}
\end{table}

The architectures of the models used in the experiments are detailed in Table \ref{tab:models}. 
Special attention is given to the ensemble models, which combine predictions from multiple models 
to enhance overall performance.\\
\noindent
All MLP\_Ensemble models consist of the individual models listed in their architecture name. 
These models’ predictions are combined using a soft voting strategy, where the final prediction is 
determined by the argmax of the sum of the predicted probabilities from each model. This approach is 
effective when the models are well-calibrated and exhibit complementary strengths and weaknesses.\\
\noindent
The ALL\_Ensemble model aggregates the predictions of all individual models using a majority vote strategy. 
In contrast, the CB\_ALL\_Ensemble model also considers all individual models but uses a CatBoost model to 
aggregate the predictions. This allows for a more flexible voting strategy, potentially leading to improved 
performance.