\vfill
\section{Introduction}
\firstword{H}{eart} diseases are the leading cause of death worldwide, accounting for 18.5 million deaths in 2019, as reported
by Max Roser in World In Data \citeyear{owid-causes-of-death-treemap} \cite{owid-causes-of-death-treemap}.
This highlights the urgent need for early detection to prevent fatalities. Traditional diagnostic methods often fall short,
diagnosing heart conditions at advanced stages when treatment options are limited. Therefore, developing an automated system for early detection is crucial.\\
Recent advancements in machine learning and deep learning have shown promise in healthcare applications, including heart disease detection.
These technologies can offer accurate and early diagnosis, potentially improving patient outcomes. However, current models have limitations.
For instance, simple models like SVM, used by Zhang et al. \cite{Zhang_Han_Deng_2017} and Deng et al. \cite{Deng_Han_2016}, have low precision,
unsuitable for medical applications. Moreover, research by Alafif et al. \cite{Alafif_Boulares_Barnawi_Alafif_Althobaiti_Alferaidi_2020}
and Noman et al. \cite{Noman_Ting_Salleh_Ombao_2019} focuses on identifying whether a heartbeat is normal or abnormal without specifying the
type of disease or the location of the abnormality in the recording. Additionally, Chen et al.'s \cite{Chen_Ren_Hao_Hu_2018} model, despite its high accuracy of 0.93,
is too complex for practical usage and distinguishes only a few types of heart diseases, which is inadequate for comprehensive medical applications.\\
To address these limitations, we propose a model that effectively identifies various heart diseases, including cases where no heart disease is present,
while balancing model complexity with result accuracy. Considering the importance of early anomaly detection in heartbeats, we also propose a lightweight
model capable of detecting anomalies in heartbeat recordings. This approach allows individuals to use their smartphones to record their heartbeats and check for anomalies,
facilitating timely consultation with a doctor for further diagnosis.


\rowcolors{2}{blue!8}{blue!18}

\begin{table*}[ht!]
    \small
    \centering
    \begin{tabular}{|c|c|c|c|c|c|}
        \hline
        \textbf{Authors}                                                                & \textbf{Models}            & \textbf{Features}     & \textbf{Results} & \textbf{Anno} & \textbf{Dataset} \\ \hline
        W. Zhang et al \cite{Zhang_Han_Deng_2017}                                       & SVM                        & Spectrogram           & 0.76 Precision   & 2017          & N, M, EH, AR     \\ \hline
        SW. Deng et al \cite{Deng_Han_2016}                                             & SVM                        & DWT                   & 0.76 Precision   & 2016          & N, M, EH, AR     \\ \hline
        A. Raza et al \cite{Raza_Mehmood_Ullah_Ahmad_Choi_On_2019}                      & LSTM                       & 1D time series        & 0.80 Accuracy    & 2019          & N, M, ES         \\ \hline
        T. Alafif et al \cite{Alafif_Boulares_Barnawi_Alafif_Althobaiti_Alferaidi_2020} & 2D-CNN + transfer learning & MFCC                  & 0.89 Accuracy    & 2020          & N, A             \\ \hline
        Noman et al \cite{Noman_Ting_Salleh_Ombao_2019}                                 & Ensemble CNN               & 1D time series + MFCC & 0.89 Accuracy    & 2019          & N, A             \\ \hline
        Chen et al \cite{Chen_Ren_Hao_Hu_2018}                                          & 2D CNN                     & WT + Hilbert-Huang    & 0.93 Accuracy    & 2018          & N, M, ES         \\ \hline
        Our Model                                                                       & Ensemble Model (MLPs + RF) & MFCC + Chroma + ZCR   & 0.88 Accuracy    & 2024          & AR, M, N, EH, ES \\ \hline
    \end{tabular}
    \caption{Comparison of different models for classification. \textbf{Legend:} N: Normal, M: Murmur, EH: Extra Heartbeat, AR: Artifact, ES: Extra systoles, A: Abnormal}
\end{table*}